\documentclass[12pt,a4paper]{report}
\usepackage{graphicx}
\usepackage[a4paper, top=0.75in, bottom=0.75in, left=1.25in, right=1in, headheight=15pt]{geometry}
\usepackage{fancyhdr}
\usepackage{hyperref}
\usepackage{listings}
\usepackage{algorithm2e}
\usepackage{color}
\usepackage{xcolor}
\usepackage{xtab,booktabs}
\usepackage{setspace}
\usepackage{amsmath}
\usepackage{float}
\usepackage{titlesec}
\usepackage{tocloft}
\usepackage{longtable}
\usepackage{array}
\usepackage{multirow}
\usepackage{caption}
\usepackage{subcaption}
\usepackage{amssymb}  % For checkmark symbol
\usepackage{times}    % Times New Roman font

% Handle missing images gracefully
\usepackage{ifthen}
\newcommand{\includelogo}[2]{%
  \IfFileExists{#1}{\includegraphics[width=#2]{#1}}{\fbox{\parbox{#2}{\centering\vspace{0.5cm}\textit{Logo Placeholder}\vspace{0.5cm}}}}%
}

% Line spacing - 1.5 as per guidelines
\setstretch{1.5}

% Allow small overfull boxes without warnings (up to 25pt)
\hfuzz=25pt

% Chapter and section formatting as per guidelines
% Chapter title: 16pt, Chapter name: 18pt centered
\titleformat{\chapter}[display]
{\normalfont\fontsize{18}{22}\bfseries\centering}
{\chaptertitlename\ \thechapter}{20pt}{\fontsize{18}{22}\bfseries}
\titlespacing*{\chapter}{0pt}{-20pt}{40pt}

% Section: 16pt, Subsection: 14pt
\titleformat{\section}{\normalfont\fontsize{16}{20}\bfseries}{\thesection}{1em}{}
\titleformat{\subsection}{\normalfont\fontsize{14}{18}\bfseries}{\thesubsection}{1em}{}

% Header/Footer setup
\pagestyle{fancy}
\fancyhf{}
\fancyhead[L]{\leftmark}
\fancyhead[R]{\thepage}
\renewcommand{\headrulewidth}{0.4pt}

% Hyperlink setup
\hypersetup{
    colorlinks=true,
    linkcolor=black,
    filecolor=magenta,
    urlcolor=blue,
    citecolor=blue
}

\begin{document}

% ============================================
% TITLE PAGE
% ============================================
\begin{titlepage}
\begin{center}
\textbf{\large VISVESVARAYA TECHNOLOGICAL UNIVERSITY}\\
\textbf{BELAGAVI}\\[1cm]

\textbf{\Large Project Report on}\\[0.5cm]
\textbf{\LARGE ``PRESENTATION HUB''}\\[1cm]

\textit{Submitted in the partial fulfillment for the requirements of the degree of}\\[0.5cm]
\textbf{\large BACHELOR OF ENGINEERING}\\
\textbf{IN}\\
\textbf{\large COMPUTER SCIENCE AND ENGINEERING}\\[1cm]

\textit{Submitted By}\\[0.3cm]
\begin{tabular}{ll}
\textbf{HIMANSHU LODHA} & \textbf{1BY22CS083}\\
\textbf{K AMRUTHA KAMATH} & \textbf{1BY22CS095}\\
\textbf{ADITYA RAJ} & \textbf{1BY22CS210}\\
\end{tabular}\\[1cm]

\textit{Under the guidance of}\\[0.3cm]
\textbf{DR. SANJAY H A}\\
\textbf{PRINCIPAL}\\
\textbf{BMSIT\&M}\\[1cm]

\textbf{\large DEPARTMENT OF COMPUTER SCIENCE AND ENGINEERING}\\[0.5cm]

\includelogo{media/bmsit_logo.png}{3cm}\\[0.5cm]

\textbf{\large BMS INSTITUTE OF TECHNOLOGY \& MANAGEMENT}\\
\textbf{YELAHANKA, BENGALURU - 560064}\\[0.5cm]

\textbf{2024-2025}

\end{center}
\end{titlepage}

% ============================================
% CERTIFICATE PAGE
% ============================================
\newpage
\thispagestyle{empty}
\begin{center}
\textbf{\large VISVESVARAYA TECHNOLOGICAL UNIVERSITY}\\
\textbf{BELAGAVI}\\[0.2cm]
\textbf{\large BMS INSTITUTE OF TECHNOLOGY AND MANAGEMENT}\\
\textbf{YELAHANKA, BENGALURU -- 560064}\\[0.3cm]

\textbf{\large DEPARTMENT OF COMPUTER SCIENCE AND ENGINEERING}\\[0.5cm]

\textbf{\Large CERTIFICATE}\\[0.5cm]
\end{center}

\noindent Certified that the project work entitled \textbf{``PRESENTATION HUB''} carried out by \textbf{Himanshu Lodha (1BY22CS083)}, \textbf{K Amrutha Kamath (1BY22CS095)}, \textbf{Aditya Raj (1BY22CS210)}, bonafide students of BMSIT\&M, in partial fulfillment for the award of Bachelor of Engineering in Computer Science and Engineering of Visvesvaraya Technological University, Belagavi during the year 2024-2025. It is certified that all corrections/suggestions indicated for Internal Assessment have been incorporated in the Report. The project report has been approved as it satisfies the academic requirements in respect of Project work prescribed for the said Degree.\\[0.8cm]

\noindent
\begin{minipage}[t]{0.48\textwidth}
\textbf{Dr. Sanjay H A}\\
Project Guide\\
Dept. of CSE, BMSIT\&M
\end{minipage}
\hfill
\begin{minipage}[t]{0.48\textwidth}
\raggedleft
\textbf{Dr. Satish Kumar T}\\
HOD, Dept. of CSE\\
BMSIT\&M
\end{minipage}\\[1cm]

\begin{center}
\textbf{Dr. Sanjay H A}\\
Principal, BMSIT\&M
\end{center}

\vspace{0.5cm}

\noindent\textbf{External Viva}\\[0.2cm]
\begin{tabular}{|p{6cm}|p{5cm}|}
\hline
\textbf{Name of the Examiners} & \textbf{Signature with Date}\\
\hline
1. & \\[0.5cm]
\hline
2. & \\[0.5cm]
\hline
\end{tabular}

% ============================================
% DECLARATION
% ============================================
\newpage
\pagenumbering{roman}
\setcounter{page}{1}
\thispagestyle{empty}
\begin{center}
\textbf{\Large DECLARATION}\\[1cm]
\end{center}

\noindent We declare that this project report titled \textbf{``PRESENTATION HUB''} submitted in partial fulfillment of the degree of B.Tech in Computer Science and Engineering is a record of original work carried out by us under the supervision of \textbf{Dr. Sanjay H A}, and has not formed the basis for the award of any other degree, in this or any other Institution or University. In keeping with the ethical practice in reporting scientific information, due acknowledgements have been made wherever the findings of others have been cited.\\[1.5cm]

\begin{center}
\textbf{Signature of the student(s)}\\[0.5cm]
\begin{tabular}{|l|l|l|}
\hline
\textbf{Name} & \textbf{USN} & \textbf{Signature}\\
\hline
Himanshu Lodha & 1BY22CS083 & \\[0.4cm]
\hline
K Amrutha Kamath & 1BY22CS095 & \\[0.4cßm]
\hline
Aditya Raj & 1BY22CS210 & \\[0.4cm]
\hline
\end{tabular}
\end{center}

% ============================================
% ACKNOWLEDGEMENT
% ============================================
\newpage
\thispagestyle{empty}
\begin{center}
\textbf{\Large ACKNOWLEDGEMENT}\\[1cm]
\end{center}

\noindent We are pleased to present this project report upon its successful completion. This project would not have been possible without the guidance, assistance, and suggestions of many individuals. We would like to express our deep sense of gratitude to each and every person who has contributed to making this project a success.\\[0.5cm]

\noindent First and foremost, we extend our heartfelt thanks to \textbf{Dr. Sanjay H A}, Principal, BMS Institute of Technology \& Management, for his constant encouragement and inspiration in undertaking this project.\\[0.5cm]

\noindent We also express our sincere gratitude to \textbf{Dr. Satish Kumar T}, Head of the Department, Computer Science and Engineering, BMS Institute of Technology \& Management, for his unwavering support and motivation throughout this endeavor.\\[0.5cm]

\noindent We express our gratitude to \textbf{Dr. Shobha M}, Associate Professor and Cluster Head, Department of Computer Science and Engineering, for her constant encouragement and valuable guidance.\\[0.5cm]

\noindent We are grateful to our project guide, \textbf{Dr. Sanjay H A}, for his invaluable encouragement and guidance throughout the project.\\[0.5cm]

\noindent Special thanks are due to all the staff members of the Computer Science and Engineering Department for their help and cooperation.\\[0.5cm]

\noindent Finally, we would like to express our gratitude to our parents and friends for their unwavering encouragement and support throughout the duration of this project.\\[1cm]

\begin{flushright}
\textbf{Himanshu Lodha}\\
\textbf{K Amrutha Kamath}\\
\textbf{Aditya Raj}
\end{flushright}

% ============================================
% ABSTRACT
% ============================================
\newpage
\thispagestyle{empty}
\begin{center}
\textbf{\Large ABSTRACT}\\[1cm]
\end{center}

\noindent This project presents Presentation Hub, an AI-powered platform designed to automate the creation of professional presentations from simple text prompts. The system leverages the OpenAI API to generate structured slide content, titles, summaries, and visual suggestions within seconds. Built with modern web technologies including Next.js and React, it provides a responsive and interactive user interface.\\[0.5cm]

\noindent The platform integrates Clerk for secure user authentication and Lemon Squeezy for subscription-based payment processing. Data persistence is handled through Prisma ORM with Neon PostgreSQL database. The frontend employs Zustand for state management and Shadcn UI components for a polished user experience.\\[0.5cm]

\noindent Testing and usage of the platform indicate high reliability, efficiency, and scalability. The system successfully demonstrates how generative AI can enhance creativity, reduce workload, and improve productivity in presentation creation across multiple domains including education, business, and professional settings.

% ============================================
% TABLE OF CONTENTS
% ============================================
\newpage
\tableofcontents

% ============================================
% LIST OF FIGURES
% ============================================
\newpage
\listoffigures

% ============================================
% LIST OF TABLES
% ============================================
\newpage
\listoftables

% Set page numbering
\newpage
\pagenumbering{arabic}
\setcounter{page}{1}

% ============================================
% CHAPTER 1: INTRODUCTION
% ============================================
\chapter{Introduction}

\section{Background}
Creating visually appealing and well-structured presentations has always been a time-consuming task, especially for students, educators, and professionals who need to present information quickly. Traditional presentation tools require users to manually design layouts, add content, choose images, and maintain consistent formatting across slides. As workload increases and deadlines become tighter, people often struggle to prepare high-quality presentations within a limited time. With the rise of AI-driven solutions, there is a growing need for tools that can automate this process and reduce the effort required to convert ideas into professional slides.

Presentation Hub addresses this need by leveraging advanced generative AI to create complete presentations from just a text prompt. Built using modern technologies like Next.js, React, OpenAI API, Clerk authentication, and Lemon Squeezy payments, the platform delivers a seamless and secure user experience. The system intelligently generates slide content, titles, summaries, and visuals, allowing users to produce ready-to-use PPT files within seconds. By integrating AI with a powerful web stack, Presentation Hub simplifies the entire presentation-creation workflow and demonstrates how automation can significantly enhance productivity and creativity.

\section{Problem Statement}
Preparing a complete and visually consistent presentation is often a tedious and time-consuming task for students, professionals, and educators. Users must manually gather information, structure the content into meaningful slides, decide on design layouts, and select visuals that match the topic. This process demands both creativity and technical presentation skills, which many users may lack. As a result, individuals often end up with poorly structured slides, mismatched designs, or rushed presentations created under tight deadlines.

Despite the availability of modern presentation tools, there is still no fast and automated solution that can convert a simple prompt or idea into a fully generated PPT. Users need a system that can instantly produce accurate content, organized slide structures, and relevant visuals without requiring extensive editing. The absence of such an AI-powered presentation generator makes it difficult for people to create high-quality presentations efficiently. Therefore, there is a need for a platform like Presentation Hub that automates the entire slide creation process and significantly reduces the effort and time required to prepare professional presentations.

\section{Problem Description}
Creating a well-structured and visually appealing presentation requires a combination of content knowledge, design skills, and a significant amount of time. Users often struggle with organizing information into meaningful slides, maintaining consistent formatting, selecting appropriate visuals, and ensuring the presentation flows smoothly. For many students, teachers, and working professionals, these tasks become overwhelming---especially when deadlines are short or when they lack strong design experience. As a result, presentations often end up rushed, inconsistent, or lacking clarity.

Although several presentation tools exist, most of them still rely heavily on manual effort. They do not offer an automated way to transform a simple text prompt into a complete, ready-to-use PowerPoint file. Users must start from scratch, design templates themselves, and manually populate content. This gap creates a strong need for an AI-powered solution that can instantly generate structured slides, relevant text, and visuals. Presentation Hub aims to fill this gap by providing an intelligent system that automates the entire presentation-creation process and significantly reduces the user's workload.

\section{Objectives}
Presentation Hub is an AI-powered platform that automatically generates complete, professional presentations from a simple text prompt. It simplifies the entire slide-creation process by producing structured content, clean layouts, and relevant visuals within seconds. The system combines OpenAI, Next.js, React, Clerk, and Lemon Squeezy to deliver a fast, secure, and seamless presentation-building experience. The objectives of this project include:

\begin{enumerate}
    \item To automate the creation of presentations using AI, reducing manual effort and saving user time.
    \item To generate structured slide content (titles, points, summaries) based on a simple text prompt.
    \item To produce professional PPT files with consistent formatting and layout.
    \item To integrate OpenAI's API for high-quality content generation and topic understanding.
    \item To offer a smooth user experience through a modern Next.js and React-based interface.
    \item To provide secure user authentication using Clerk for safe login and access control.
\end{enumerate}

\section{WPs and SDG Addressed}

\subsection{Mapping of Complex Engineering Problems with Washington Accord WPs (WP1-WP7)}

\begin{table}[H]
\centering
\caption{Washington Accord WP Mapping}
\small
\begin{tabular}{|c|p{2.5cm}|p{7cm}|c|}
\hline
\textbf{WP} & \textbf{Description} & \textbf{Competencies Addressed} & \textbf{Applicable}\\[0.2cm]
\hline
WP1 & In-depth Engineering Knowledge & Advanced algorithms, complex data structures, domain knowledge, professional-grade tools & \checkmark \\
\hline
WP2 & Technical \& Non-Technical Issues & Security-performance balance, resource evaluation, UX needs, privacy compliance & \checkmark \\
\hline
WP3 & Abstract Thinking \& Originality & Novel algorithm design, original system architecture, ML model innovations & \checkmark \\
\hline
WP4 & Design \& Development & Real-world problem solving, cutting-edge technologies, custom implementations & \checkmark \\
\hline
WP5 & Use of Modern Tools & Custom security mechanisms, database optimization, automated testing & \checkmark \\
\hline
WP6 & Uncertainty \& Ambiguity & User requirement analysis, role-based access, stakeholder management & \checkmark \\
\hline
WP7 & Multidisciplinary Factors & Layered architecture, modular design, DevOps integration, reliability testing & \checkmark \\
\hline
\end{tabular}
\label{table:wp_mapping}
\end{table}

\begin{table}[H]
\centering
\caption{WP Code Mapping Summary}
\begin{tabular}{|c|c|c|}
\hline
\textbf{WP Code} & \textbf{Competencies Count} & \textbf{Level}\\
\hline
WP1 & 5 & HIGH\\
WP2 & 6 & HIGH\\
WP3 & 3 & MODERATE\\
WP4 & 5 & HIGH\\
WP5 & 4 & MODERATE\\
WP6 & 4 & MODERATE\\
WP7 & 5 & HIGH\\
\hline
\end{tabular}
\label{table:wp_mapping}
\end{table}

\noindent \textbf{Note:} Scale for mapping:
\begin{itemize}
    \item Low: 1--2 competencies matched
    \item Moderate: 3--4 competencies
    \item High: 5 or more
\end{itemize}

\noindent \textbf{Complex Engineering Project:} A project is considered complex if:
\begin{itemize}
    \item \textbf{Condition A:} At least 3 out of WP1--WP7 are High
    \item \textbf{Condition B:} At least 5 out of WP1--WP7 are Moderate or High
\end{itemize}
If both conditions are met $\rightarrow$ Complex Project.

\noindent \textbf{Conclusion:} Based on the above key indicators, the Project is considered a \textbf{Complex Project}.

\subsection{Mapping of Project with SDG Goals, Targets, and Indicators}

\begin{table}[H]
\centering
\caption{Sustainable Development Goals (SDG) Mapping}
\small
\begin{tabular}{|p{2cm}|p{2.5cm}|p{3.5cm}|p{3cm}|}
\hline
\textbf{SDG Goal} & \textbf{Target} & \textbf{Justification} & \textbf{Impact Level}\\[0.2cm]
\hline
SDG 4: Quality Education & Digital skills \& learning tools & Generates structured learning materials quickly & HIGH\\
\hline
SDG 9: Industry \& Innovation & Research \& tech capabilities & Uses AI, automation, cloud infrastructure & HIGH\\
\hline
SDG 8: Decent Work & Productivity \& innovation & Automates manual work, increases efficiency & MODERATE\\
\hline
SDG 12: Responsible Consumption & Efficient resource use & Digital creation reduces paper waste & HIGH\\
\hline
\end{tabular}
\label{table:sdg_mapping}
\end{table}

% ============================================
% CHAPTER 2: LITERATURE REVIEW
% ============================================
\chapter{Literature Review}

\section{Analysis of the Literature}
With the rapid development in AI, NLP, and ML, digital content creation has undergone a significant transformation, especially regarding automated presentation generation. Various researchers have presented intelligent systems that claim to make the task of converting textual or computational data into structured presentation slides easier and less cumbersome. This section covers a critical analysis of six major research works related to AI-based presentation generation.

\subsection{OutlineSpark (Fengjie Wang, 2024)}
This focuses directly on the generation of presentation slides from computational notebooks such as Jupyter through a well-structured outline approach. It includes the display of an overview of the notebook, an outline panel, and a slide panel. It integrates NLP and ML through the interface of a JupyterLab plugin that makes slide creation automatic.

\textbf{Strengths:}
\begin{itemize}
    \item Reduces manual effort to convert notebooks into slides
    \item Well-organized content extraction
    \item Improves researcher productivity
\end{itemize}

\textbf{Limitations:}
\begin{itemize}
    \item Only accepts notebook-based input
    \item No advanced visual editing tools
    \item Lacks collaboration and cloud storage
    \item Limited UI experience
\end{itemize}

\subsection{AI pptX (Vineeth Ravi, 2019)}
It introduces a conversational AI-based document generation system in which users issue natural-language commands that are translated into actionable ``skills.'' It makes use of a CRF-based parser and a Robust Knowledge Base (RKB) for continuous learning.

\textbf{Strengths:}
\begin{itemize}
    \item Supports voice/text commands
    \item Continuous learning from user behavior
    \item Reduces manual command effort
\end{itemize}

\textbf{Limitations:}
\begin{itemize}
    \item Back-end driven with no interactive UI
    \item No drag-and-drop editor
    \item No project dashboard
    \item No authentication or payment system
\end{itemize}

\subsection{PPTAgent (Hao Zheng, 2025)}
PPTAgent analyses existing presentations to learn structural patterns through a two-stage AI pipeline, thereby automatically generating new presentations.

\textbf{Strengths:}
\begin{itemize}
    \item High-quality slide structuring
    \item Generating content and layout automatically
    \item Strong evaluation framework
\end{itemize}

\textbf{Limitations:}
\begin{itemize}
    \item No real-time editing
    \item No user customization
    \item No theming features
    \item No authentication and monetization
\end{itemize}

\subsection{Presentify (Atul Shreewastav, 2024)}
Presentify automates the generation of presentation slides based on academic research using a fine-tuned T5 Transformer.

\textbf{Strengths:}
\begin{itemize}
    \item Accurate academic summarization
    \item Structured slide generation
    \item Useful for educators and researchers
\end{itemize}

\textbf{Limitations:}
\begin{itemize}
    \item Restricted to academic content
    \item No interactive editing
    \item No theming options
    \item No authentication system
\end{itemize}

\subsection{D2S -- Document to Slide (Edward Sun -- IBM, 2021)}
D2S uses a two-step summarization in which the model first uses slide titles to retrieve relevant text, and then long-form QA techniques to generate the bullet points.

\textbf{Strengths:}
\begin{itemize}
    \item High-quality Summarization
    \item Benchmark dataset contribution
    \item Performs better than traditional summarization
\end{itemize}

\textbf{Limitations:}
\begin{itemize}
    \item Focused only on academic content
    \item No real-time editing
    \item No design customization
    \item No frontend interface
\end{itemize}

\subsection{AutoPresent (Jiaxin Ge, 2025)}
AutoPresent creates presentation slides from natural language descriptions and introduces the SlidesBench dataset.

\textbf{Strengths:}
\begin{itemize}
    \item Generates structured slides from prompts
    \item Self-improving design mechanism
    \item Large benchmark dataset
\end{itemize}

\textbf{Limitations:}
\begin{itemize}
    \item No real-time editing
    \item No authentication or user management
    \item Limited design customization
    \item No monetization capabilities
\end{itemize}

\section{Summary of the Review}

\begin{table}[H]
\centering
\caption{Summary of Literature Review}
\small
\begin{tabular}{|l|c|l|l|l|}
\hline
\textbf{Research Work} & \textbf{Year} & \textbf{Technology} & \textbf{Focus} & \textbf{Limitations}\\[0.1cm]
\hline
OutlineSpark & 2024 & NLP, ML & Notebook-to-slide & No UI editor\\
AI pptX & 2019 & CRF, NLP & Command-based & No frontend\\
PPTAgent & 2025 & Two-stage AI & Pattern-based & No real-time editing\\
Presentify & 2024 & T5 Transformer & Research-to-PPT & Domain-specific\\
D2S & 2021 & Long-form QA & Document-to-slides & No visual editor\\
AutoPresent & 2025 & LLaMA & Prompt-to-slides & No frontend\\
\hline
\end{tabular}
\label{table:literature_summary}
\end{table}

\section{Implications}
The literature reviewed clearly shows that AI-based presentation generation is an emerging and highly impactful research area. Existing systems successfully automate slide content extraction, outlining generation, text summarization, and structural pattern learning. However, nearly all of these systems lack important real-world usability features including real-time editable slide interface, drag-and-drop editing, theme selection, cloud storage, user authentication, and PPT sharing options.

\section{Existing System}
Most current AI-based presentation generation systems are backend tools or research prototypes rather than full-fledged applications. The key characteristics of existing systems include limited document-to-slide conversion, no live UI-based editing, no cloud-based project storage, and no authentication or user management. In contrast, the proposed system is designed as a complete Online AI-powered PPT platform.

% ============================================
% CHAPTER 3: SYSTEM DESIGN
% ============================================
\chapter{System Design}

\section{Architecture Diagram}

\begin{figure}[H]
\centering
\fbox{\parbox{0.9\textwidth}{\centering\vspace{2cm}\textit{[Architecture Diagram Placeholder - Insert Figure 3.1]}\vspace{2cm}}}
\caption{Architecture Diagram}
\label{fig:architecture}
\end{figure}

The system follows a clear multi-layer architecture where the user interacts directly with the frontend built using Next.js 15 and React. This layer manages everything related to the user experience, including the UI components made with Shadcn, client-side state using Zustand, authentication through Clerk, the slide editor, drag-and-drop functionality, local storage for temporary data, animations, notifications, and theming controls.

The backend runs through Next.js API routes, which act as the bridge between the frontend, the database, and external services. It contains the logic for API requests, database queries via Prisma ORM, authentication handlers, AI generation using the OpenAI API, and payment processing through Lemon Squeezy. All persistent data is saved in a Neon PostgreSQL database.

\section{Data Flow Diagram}

\begin{figure}[H]
\centering
\fbox{\parbox{0.9\textwidth}{\centering\vspace{3cm}\textit{[Data Flow Diagram Placeholder - Insert Figure 3.2]}\vspace{3cm}}}
\caption{Data Flow Diagram}
\label{fig:dfd}
\end{figure}

The system begins in the \textbf{Idle} state where the interface waits for the user to sign in. Once logged in, the system transitions to the \textbf{Authenticated} state. From here, the user enters the \textbf{Input} state where they provide the topic for the AI-generated presentation. This data moves to the \textbf{Prompts Generate} state. After AI returns results, the system moves into the \textbf{Editor} state for customization. Finally, the \textbf{Export} state converts the presentation into downloadable formats.

\section{Algorithm Diagram}

\begin{figure}[H]
\centering
\fbox{\parbox{0.9\textwidth}{\centering\vspace{2cm}\textit{[Algorithm Diagram Placeholder - Insert Figure 3.3]}\vspace{2cm}}}
\caption{Algorithm Diagram}
\label{fig:algorithm}
\end{figure}

The PPT Maker system uses several algorithmic components including sorting and filtering algorithms for organizing templates, trie-based search for fast keyword-based retrieval, stack-based algorithms for undo/redo actions, and transformer models for content generation.

% ============================================
% CHAPTER 4: SYSTEM REQUIREMENTS
% ============================================
\chapter{System Requirements Specifications}

\section{Hardware Requirements}
\begin{itemize}
    \item \textbf{Development Machine}: Intel Core i5 or equivalent, 8GB RAM minimum
    \item \textbf{Storage}: 256GB SSD with at least 10GB free space
    \item \textbf{Network}: Stable internet connection for API calls and deployment
    \item \textbf{Display}: 1920x1080 resolution for development and testing
\end{itemize}

\section{Software Requirements}
\begin{itemize}
    \item \textbf{Operating System}: Windows 10/11, macOS 12+, or Ubuntu 20.04+
    \item \textbf{Runtime}: Node.js v18+ with npm or Bun package manager
    \item \textbf{IDE}: Visual Studio Code with ESLint and Prettier extensions
    \item \textbf{Browser}: Chrome, Firefox, or Edge (latest versions)
    \item \textbf{Database}: PostgreSQL (Neon cloud database)
    \item \textbf{Version Control}: Git with GitHub repository
\end{itemize}

\section{Functional Requirements}
\begin{enumerate}
    \item \textbf{FR-01: User Registration} -- Users shall be able to create accounts using email or OAuth providers
    \item \textbf{FR-02: User Authentication} -- System shall authenticate users securely using Clerk
    \item \textbf{FR-03: Presentation Creation} -- Users shall generate presentations from text prompts using AI
    \item \textbf{FR-04: Slide Editing} -- Users shall edit slides with drag-and-drop functionality
    \item \textbf{FR-05: Theme Selection} -- Users shall apply and switch between different visual themes
    \item \textbf{FR-06: Presentation Storage} -- System shall save presentations to cloud database
    \item \textbf{FR-07: PPT Export} -- Users shall download presentations as PowerPoint files
    \item \textbf{FR-08: Subscription Management} -- System shall handle premium subscriptions via Lemon Squeezy
\end{enumerate}

\section{Non-Functional Requirements}
\begin{enumerate}
    \item \textbf{NFR-01: Performance} -- API response time shall be under 3 seconds for AI generation
    \item \textbf{NFR-02: Scalability} -- System shall support 100+ concurrent users
    \item \textbf{NFR-03: Security} -- All data transmission shall use HTTPS encryption
    \item \textbf{NFR-04: Availability} -- System uptime shall be 99.5\% or higher
    \item \textbf{NFR-05: Usability} -- Interface shall be intuitive requiring no training
    \item \textbf{NFR-06: Maintainability} -- Code shall follow modular architecture patterns
    \item \textbf{NFR-07: Compatibility} -- Application shall work on all modern browsers
    \item \textbf{NFR-08: Data Integrity} -- Database transactions shall maintain ACID properties
\end{enumerate}

% ============================================
% CHAPTER 5: IMPLEMENTATION
% ============================================
\chapter{Implementation}

\section{Software Frameworks and Tools Used}
\begin{enumerate}
    \item \textbf{Next.js} -- for building the frontend with server-side rendering
    \item \textbf{React} -- for creating interactive UI components
    \item \textbf{Tailwind CSS} -- for modern, responsive styling
    \item \textbf{Zustand} -- for client-side state management
    \item \textbf{Clerk} -- for secure user authentication
    \item \textbf{Lemon Squeezy} -- for managing subscriptions and payments
    \item \textbf{OpenAI API} -- for AI-powered slide content generation
    \item \textbf{Prisma ORM} -- for database communication
    \item \textbf{PostgreSQL (Neon)} -- as the relational database
    \item \textbf{VS Code and GitHub} -- for development and version control
\end{enumerate}

\section{Technologies Used}
The PPT Maker project employs a range of modern technologies to deliver a seamless and intelligent presentation creation experience. It uses Next.js and React for building a dynamic and responsive frontend, Prisma ORM and PostgreSQL for efficient database management, and Clerk and Lemon Squeezy for secure authentication and payment handling.

\section{Algorithms Used}
\begin{itemize}
    \item \textbf{Sorting and filtering algorithms} -- organize templates and slides
    \item \textbf{Trie-based search} -- fast keyword-based lookup
    \item \textbf{Stack-based algorithms} -- undo and redo operations
    \item \textbf{Transformer-based models} -- generate slide content from prompts
    \item \textbf{Text summarization algorithms} -- condense input into slide content
    \item \textbf{Semantic similarity techniques} -- suggest related templates
\end{itemize}

% ============================================
% CHAPTER 6: TESTING AND RESULTS
% ============================================
\chapter{Testing and Results}

\section{Appropriate Testing Methods}

\subsection{Unit Testing}
Unit testing involves testing individual components to ensure each part works correctly in isolation.

\subsection{Integration Testing}
Integration testing verifies that different modules work together correctly.

\subsection{System Testing}
System testing evaluates the complete application to ensure it meets requirements.

\subsection{Functional Testing}
Functional testing assesses whether the software performs its intended functions correctly.

\subsection{Performance Testing}
Performance testing measures how the system performs under various loads.

\subsection{Security Testing}
Security testing identifies vulnerabilities and ensures data protection.

\section{Screenshots}

\begin{figure}[H]
\centering
\fbox{\parbox{0.9\textwidth}{\centering\vspace{2cm}\textit{[Home Page Screenshot]}\vspace{2cm}}}
\caption{Home Page}
\label{fig:home}
\end{figure}

\begin{figure}[H]
\centering
\fbox{\parbox{0.9\textwidth}{\centering\vspace{2cm}\textit{[New Project Page Screenshot]}\vspace{2cm}}}
\caption{New Project Page}
\label{fig:newproject}
\end{figure}

\begin{figure}[H]
\centering
\fbox{\parbox{0.9\textwidth}{\centering\vspace{2cm}\textit{[Editing Page Screenshot]}\vspace{2cm}}}
\caption{Editing Page}
\label{fig:editing}
\end{figure}

\section{Testing Metrics with Results}

\begin{table}[H]
\centering
\caption{Testing Results Summary}
\begin{tabular}{|l|c|c|c|}
\hline
\textbf{Test Type} & \textbf{Test Cases} & \textbf{Passed} & \textbf{Pass Rate}\\[0.1cm]
\hline
Unit Testing & 50 & 48 & 96\%\\
Integration Testing & 25 & 24 & 96\%\\
System Testing & 15 & 14 & 93\%\\
Functional Testing & 30 & 29 & 97\%\\
\hline
\textbf{Total} & \textbf{120} & \textbf{115} & \textbf{95.8\%}\\
\hline
\end{tabular}
\label{table:testing_metrics}
\end{table}

% ============================================
% CHAPTER 7: RESULTS
% ============================================
\chapter{Results}

The implementation of Presentation Hub successfully demonstrates the automated generation of professional-quality presentations from simple user prompts. Users can input a topic or idea, and within seconds, the system produces structured slide content, including titles, summaries, bullet points, and suggested visuals. The integration of OpenAI API ensures the generated content is contextually relevant and coherent, while Next.js and React provide a responsive and interactive user interface.

Testing and usage of the platform indicate high reliability, efficiency, and scalability. Presentations generated are consistent in formatting, visually appealing, and ready for immediate download as PPT files.

\section{Comparison with Other Models}

\begin{table}[H]
\centering
\caption{Feature Comparison with Existing Systems}
\small
\begin{tabular}{|l|c|c|c|c|}
\hline
\textbf{Feature} & \textbf{Presentation Hub} & \textbf{OutlineSpark} & \textbf{AI pptX} & \textbf{PPTAgent}\\[0.1cm]
\hline
AI Content Generation & \checkmark & \checkmark & \checkmark & \checkmark\\
Real-time Editing & \checkmark & $\times$ & $\times$ & $\times$\\
Theme Customization & \checkmark & $\times$ & $\times$ & $\times$\\
Cloud Storage & \checkmark & $\times$ & $\times$ & $\times$\\
User Authentication & \checkmark & $\times$ & $\times$ & $\times$\\
Payment Integration & \checkmark & $\times$ & $\times$ & $\times$\\
\hline
\end{tabular}
\label{table:comparison}
\end{table}

% ============================================
% CHAPTER 8: CONCLUSION AND FUTURE WORK
% ============================================
\chapter{Conclusion and Future Work}

\section{Summary}
Presentation Hub is an AI-powered platform designed to automate the creation of professional presentations from simple text prompts. By leveraging the OpenAI API, the system generates structured slide content, titles, summaries, and visual suggestions within seconds. Built with modern web technologies like Next.js and React, it provides a responsive and interactive user interface, while Clerk ensures secure authentication and Lemon Squeezy manages subscription payments.

\section{Limitations}
\begin{itemize}
    \item Output quality depends on the clarity of the user's input prompt
    \item Requires a stable internet connection for AI processing
    \item Limited customization options compared to manual tools
    \item Advanced animations are not fully automated
    \item AI-generated content may occasionally need manual proofreading
\end{itemize}

\section{Improvements}
\begin{enumerate}
    \item Include more advanced design templates and themes
    \item Integration of interactive elements like charts and graphs
    \item Adding multilingual support
    \item Implementing offline functionality
    \item Incorporating AI-assisted proofreading
    \item Improving collaboration features
\end{enumerate}

% ============================================
% REFERENCES
% ============================================
\chapter*{References}
\addcontentsline{toc}{chapter}{References}

\begin{enumerate}
    \item[{[1]}] Fengjie Wang. ``OutlineSpark: Igniting AI-powered Presentation Slides Creation from Computational Notebooks through Outlines,'' ACM CHI Conference on Human Factors in Computing Systems, 2024.
    
    \item[{[2]}] Vineeth Ravi. ``AI pptX: Robust Continuous Learning for Document Generation with AI Insights,'' IEEE International Conference, 2019.
    
    \item[{[3]}] Hao Zheng. ``PPTAgent: Generating and Evaluating Presentations Beyond Text-to-Slides,'' arXiv preprint arXiv:2501.03936, 2025.
    
    \item[{[4]}] Atul Shreewastav. ``Presentify: Automated Academic Presentation Generation using T5 Transformer,'' 2024.
    
    \item[{[5]}] Edward Sun (IBM). ``D2S: Document to Slide Generation via Query-Based Summarization,'' 2021.
    
    \item[{[6]}] Jiaxin Ge. ``AutoPresent: Generating Presentations from Natural Language with SlidesBench,'' 2025.
    
    \item[{[7]}] A. Vaswani et al. ``Attention is All You Need,'' Advances in Neural Information Processing Systems, 2017.
    
    \item[{[8]}] T. Brown et al. ``Language Models are Few-Shot Learners,'' Advances in Neural Information Processing Systems, 2020.
    
    \item[{[9]}] Next.js Documentation, Vercel Inc. \url{https://nextjs.org/docs}
    
    \item[{[10]}] OpenAI API Documentation. \url{https://platform.openai.com/docs}
\end{enumerate}

\end{document}
