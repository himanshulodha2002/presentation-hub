\documentclass[12pt,a4paper]{report}
\usepackage{graphicx}
\usepackage[a4paper, top=0.75in, bottom=0.75in, left=1.25in, right=1in, headheight=15pt]{geometry}
\usepackage{fancyhdr}
\usepackage{hyperref}

% Disable automatic hyphenation
\tolerance=1
\emergencystretch=\maxdimen
\hyphenpenalty=10000
\hbadness=10000
\usepackage{listings}
\usepackage{algorithm2e}
\usepackage{color}
\usepackage{xcolor}
\usepackage{xtab,booktabs}
\usepackage{setspace}
\usepackage{amsmath}
\usepackage{float}
\usepackage{titlesec}
\usepackage{tocloft}
\usepackage{longtable}
\usepackage{array}
\usepackage{multirow}
\usepackage{caption}
\usepackage{subcaption}
\usepackage{amssymb}  % For checkmark symbol
\usepackage{times}    % Times New Roman font
\usepackage{indentfirst}
\usepackage{tikz}
\usepackage{enumitem} % For compact lists in tables
\usetikzlibrary{calc}

% Handle missing images gracefully
\usepackage{ifthen}
\newcommand{\includelogo}[2]{%
  \IfFileExists{#1}{\includegraphics[width=#2]{#1}}{\fbox{\parbox{#2}{\centering\vspace{0.5cm}\textit{Logo Placeholder}\vspace{0.5cm}}}}%
}

% Line spacing - 1.5 as per guidelines
\setstretch{1.5}

% Allow small overfull boxes without warnings (up to 25pt)
\hfuzz=25pt

% Chapter and section formatting as per guidelines
% Chapter title: 16pt, Chapter name: 18pt centered
\titleformat{\chapter}[display]
{\normalfont\fontsize{18}{22}\bfseries\centering}
{\chaptertitlename\ \thechapter}{18pt}{\fontsize{18}{22}\bfseries}
\titlespacing*{\chapter}{0pt}{-20pt}{30pt}

% Reduce space after TOC/LOF/LOT titles
\renewcommand{\cfttoctitlefont}{\hfill\huge\bfseries}
\renewcommand{\cftaftertoctitle}{\hfill}
\renewcommand{\cftloftitlefont}{\hfill\huge\bfseries}
\renewcommand{\cftafterloftitle}{\hfill}
\renewcommand{\cftlottitlefont}{\hfill\huge\bfseries}
\renewcommand{\cftafterlottitle}{\hfill}
\setlength{\cftaftertoctitleskip}{20pt}
\setlength{\cftafterloftitleskip}{20pt}
\setlength{\cftafterlottitleskip}{20pt}

% Section: 16pt, Subsection: 14pt
\titleformat{\section}{\normalfont\fontsize{16}{20}\bfseries}{\thesection}{1em}{}
\titleformat{\subsection}{\normalfont\fontsize{14}{18}\bfseries}{\thesubsection}{1em}{}

% Header/Footer setup
\pagestyle{fancy}
\fancyhf{}
\fancyhead[L]{\leftmark}
\fancyhead[R]{\thepage}
\renewcommand{\headrulewidth}{0.4pt}

% Hyperlink setup
\hypersetup{
    colorlinks=true,
    linkcolor=black,
    filecolor=magenta,
    urlcolor=blue,
    citecolor=blue
}

\begin{document}

% ============================================
% TITLE PAGE
% ============================================
\begin{titlepage}

% Border around title page
\begin{tikzpicture}[remember picture, overlay]
    \draw[line width=1.5pt]
    ($(current page.north west) + (1.5cm,-1.5cm)$)
    rectangle
    ($(current page.south east) + (-1.5cm,1.5cm)$);
\end{tikzpicture}

\begin{center}
\textbf{\large VISVESVARAYA TECHNOLOGICAL UNIVERSITY}\\
\textbf{\small Jnana Sangama, Belagavi, Karnataka - 590018}\\[0.4cm]

\includelogo{media/vtu_logo.png}{2.5cm}\\[0.4cm]

\textbf{\large A Project Report on}\\[0.3cm]
\textbf{\LARGE ``PRESENTATION HUB''}\\[0.5cm]

\textit{Submitted in partial fulfilment of the requirements for the conferment of degree of}\\[0.3cm]
\textbf{\large BACHELOR OF ENGINEERING}\\
\textit{in}\\
\textbf{\large COMPUTER SCIENCE AND ENGINEERING}\\[0.5cm]

\textit{by}\\[0.3cm]
\textbf{HIMANSHU LODHA (1BY22CS083)}\\
\textbf{K AMRUTHA KAMATH (1BY22CS095)}\\
\textbf{ADITYA RAJ (1BY22CS210)}\\[0.5cm]

\textit{Under the Guidance of}\\[0.2cm]
\textbf{Dr. Sanjay H A}\\
\textit{Principal, BMSIT\&M}\\[0.5cm]

\textbf{Department of Computer Science and Engineering}\\[0.2cm]

\includelogo{media/bmsit_logo.png}{2.5cm}\\[0.2cm]

\textbf{\large BMS INSTITUTE OF TECHNOLOGY AND MANAGEMENT}\\
\small (Autonomous Institute under VTU, Belagavi, Karnataka - 590018)\\
\small Yelahanka, Bengaluru, Karnataka - 560064\\[0.5cm]

\textbf{2025--2026}

\end{center}
\end{titlepage}

% ============================================
% CERTIFICATE PAGE
% ============================================
\newpage
\thispagestyle{empty}
\begin{center}
\textbf{\large BMS INSTITUTE OF TECHNOLOGY AND MANAGEMENT}\\
\small (Autonomous Institute under VTU, Belagavi, Karnataka - 590018)\\
\small Yelahanka, Bengaluru, Karnataka - 560064\\[0.5cm]

\includelogo{media/bmsit_logo.png}{2.5cm}\\[0.3cm]

\textbf{\Large CERTIFICATE}\\[0.5cm]
\end{center}

\noindent This is to certify that the project entitled \textbf{``PRESENTATION HUB''} is a bona fide work carried out by \textbf{Himanshu Lodha (1BY22CS083)}, \textbf{K Amrutha Kamath (1BY22CS095)}, and \textbf{Aditya Raj (1BY22CS210)} in partial fulfilment for the award of ``BACHELOR OF ENGINEERING'' in ``Computer Science and Engineering'' of the Visvesvaraya Technological University, Belagavi, during the year 2025--2026. It is certified that all corrections and suggestions indicated for internal assessment have been incorporated in the report. The project report has been approved as it satisfies the academic requirements in respect to work for the BE degree.

\vspace{0.8cm}

% Signature Block - Row 1: Two signatures on left and right
\noindent
\begin{minipage}[t]{0.45\textwidth}
\centering
\textbf{Dr. Shobha M}\\
Associate Professor and Cluster Head\\
Department of CSE-2
\end{minipage}
\hfill
\begin{minipage}[t]{0.45\textwidth}
\centering
\textbf{Dr. Satish Kumar T}\\
Associate Professor and Head of Department\\
Department of CSE
\end{minipage}

\vspace{1.5cm}

% Row 2: Principal centered
\begin{center}
\textbf{Dr. Sanjay H A}\\
Professor \& Principal (Project Guide)\\
BMSIT\&M
\end{center}

\vspace{0.8cm}

\noindent\textbf{Name of the Examiners} \hfill \textbf{Signature with Date}\\[0.5cm]
1. .............................................................. \hfill ....................................\\[0.5cm]
2. .............................................................. \hfill ....................................

% ============================================
% ACKNOWLEDGEMENT
% ============================================
\newpage
\pagenumbering{roman}
\setcounter{page}{1}
\chapter*{ACKNOWLEDGEMENT}
\thispagestyle{empty}
\addcontentsline{toc}{chapter}{Acknowledgement}

\indent We would like to express our heartfelt gratitude to everyone who has contributed to make this project a memorable experience and has inspired this work in some way.

Let us begin by expressing our gratitude to the Almighty God for the numerous blessings bestowed upon us. We are happy to present this project after completing it successfully.

This project would not have been possible without the guidance, assistance, and suggestions of many individuals. We express our deep sense of gratitude and indebtedness to each and every one who has helped us make this project a success.

We heartily thank \textbf{Dr. Sanjay H A}, Professor \& Principal, BMS Institute of Technology \& Management, who also serves as our project guide, for his constant encouragement, inspiration, and valuable guidance throughout this project.

We heartily thank \textbf{Dr. Satish Kumar T}, Associate Professor and Head of Department, Department of Computer Science and Engineering, BMS Institute of Technology \& Management for constant encouragement and inspiration in taking up this project.

We heartily thank \textbf{Dr. Shobha M}, Associate Professor and Cluster Head, BMS Institute of Technology \& Management for constant encouragement and inspiration in taking up this project.

Special thanks to all the staff members of the Computer Science and Engineering Department for their help and kind co-operation. Lastly, we thank our parents and friends for their encouragement and support in helping us complete this work.

\vspace{1cm}

\begin{flushright}
\textbf{HIMANSHU LODHA (1BY22CS083)}\\
\textbf{K AMRUTHA KAMATH (1BY22CS095)}\\
\textbf{ADITYA RAJ (1BY22CS210)}
\end{flushright}

% ============================================
% DECLARATION
% ============================================
\newpage
\thispagestyle{empty}
\addcontentsline{toc}{chapter}{Declaration}
\begin{center}
\textbf{\Large DECLARATION}\\[1cm]
\end{center}

\noindent We hereby declare that the project titled \textbf{``PRESENTATION HUB''} is a record of original project work under the guidance of \textbf{Dr. Sanjay H A} (Professor \& Principal, serving as Project Guide), BMS Institute of Technology \& Management, Autonomous Institute under Visvesvaraya Technological University, Belagavi, during the Academic Year 2025--2026.

We also declare that this project report has not been submitted for the award of any degree, diploma, associateship, fellowship or other title anywhere else.

\vspace{1cm}
\begin{table}[h]
\centering
\begin{tabular}{|c|c|c|}
\hline
\textbf{Name of the Student} & \textbf{USN} & \textbf{Signature} \\ \hline
\rule{0pt}{3ex} Himanshu Lodha & 1BY22CS083 &  \\ \hline
\rule{0pt}{3ex} K Amrutha Kamath & 1BY22CS095 &  \\ \hline
\rule{0pt}{3ex} Aditya Raj & 1BY22CS210 &  \\ \hline
\end{tabular}
\end{table}

% ============================================
% ABSTRACT
% ============================================
\newpage
\chapter*{ABSTRACT}
\thispagestyle{empty}
\addcontentsline{toc}{chapter}{Abstract}

\noindent This project presents Presentation Hub, an AI-powered platform designed to automate the creation of professional presentations from simple text prompts. The system leverages the OpenAI API to generate structured slide content, titles, summaries, and visual suggestions within seconds. Built with modern web technologies including Next.js and React, it provides a responsive and interactive user interface. The platform integrates Clerk for secure user authentication and Lemon Squeezy for subscription-based payment processing, while data persistence is handled through Prisma ORM with Neon PostgreSQL database. The frontend employs Zustand for state management and Shadcn UI components for a polished user experience. Testing results demonstrate high reliability and efficiency, with users reporting satisfaction with the quality of generated presentations. The platform successfully demonstrates how generative AI can enhance creativity, reduce workload, and improve productivity in presentation creation across multiple domains including education, business, and professional settings.

% ============================================
% TABLE OF CONTENTS
% ============================================
\newpage
\tableofcontents

% ============================================
% LIST OF FIGURES
% ============================================
\newpage
\listoffigures
\addcontentsline{toc}{chapter}{List of Figures}

% ============================================
% LIST OF TABLES
% ============================================
\newpage
\listoftables
\addcontentsline{toc}{chapter}{List of Tables}

% Set page numbering
\newpage
\pagenumbering{arabic}
\setcounter{page}{1}

% ============================================
% CHAPTER 1: INTRODUCTION
% ============================================
\chapter{Introduction}

\section{Background}
Creating visually appealing and well-structured presentations has always been a time-consuming task, especially for students, educators, and professionals who need to present information quickly. Traditional presentation tools require users to manually design layouts, add content, choose images, and maintain consistent formatting across slides. As workload increases and deadlines become tighter, people often struggle to prepare high-quality presentations within a limited time. With the rise of AI-driven solutions, there is a growing need for tools that can automate this process and reduce the effort required to convert ideas into professional slides.

Presentation Hub addresses this need by leveraging advanced generative AI to create complete presentations from just a text prompt. Built using modern technologies like Next.js, React, OpenAI API, Clerk authentication, and Lemon Squeezy payments, the platform delivers a seamless and secure user experience. The system intelligently generates slide content, titles, summaries, and visuals, allowing users to produce ready-to-use PPT files within seconds. By integrating AI with a powerful web stack, Presentation Hub simplifies the entire presentation-creation workflow and demonstrates how automation can significantly enhance productivity and creativity.

\section{Problem Statement}
Creating professional presentations remains a time-consuming and skill-intensive task for students, professionals, and educators. Despite the availability of modern presentation tools, there is no fast and automated solution that can convert a simple prompt into a fully generated PPT with structured content, consistent design, and relevant visuals. This gap creates a need for an AI-powered platform that automates the entire slide creation process and significantly reduces the effort required to prepare professional presentations.

\section{Problem Description}
Creating a well-structured and visually appealing presentation requires a combination of content knowledge, design skills, and a significant amount of time. Users often struggle with organizing information into meaningful slides, maintaining consistent formatting, selecting appropriate visuals, and ensuring the presentation flows smoothly. For many students, teachers, and working professionals, these tasks become overwhelming---especially when deadlines are short or when they lack strong design experience. As a result, presentations often end up rushed, inconsistent, or lacking clarity.

Although several presentation tools exist, most of them still rely heavily on manual effort. They do not offer an automated way to transform a simple text prompt into a complete, ready-to-use PowerPoint file. Users must start from scratch, design templates themselves, and manually populate content. This gap creates a strong need for an AI-powered solution that can instantly generate structured slides, relevant text, and visuals. Presentation Hub aims to fill this gap by providing an intelligent system that automates the entire presentation-creation process and significantly reduces the user's workload.

\section{Objectives}
Presentation Hub is an AI-powered platform that automatically generates complete, professional presentations from a simple text prompt. It simplifies the entire slide-creation process by producing structured content, clean layouts, and relevant visuals within seconds. The system combines OpenAI, Next.js, React, Clerk, and Lemon Squeezy to deliver a fast, secure, and seamless presentation-building experience. The objectives of this project include:

\begin{enumerate}
    \item To automate the creation of presentations using AI, reducing manual effort and saving user time.
    \item To generate structured slide content (titles, points, summaries) based on a simple text prompt.
    \item To produce professional PPT files with consistent formatting and layout.
    \item To integrate OpenAI's API for high-quality content generation and topic understanding.
    \item To offer a smooth user experience through a modern Next.js and React-based interface.
    \item To provide secure user authentication using Clerk for safe login and access control.
\end{enumerate}

\section{WPs and SDG Addressed}

\begin{center}
\textbf{\underline{Solving Complex Engineering Problems Incorporating Sustainability Goals}}\\[0.3cm]
Mapping of Complex Engineering Problems with Washington Accord WPs (WP1--WP7)
\end{center}

\begin{longtable}{|c|p{3cm}|p{7.5cm}|c|}
\hline
\textbf{WP Code} & \textbf{Description} & \textbf{Competencies} & \textbf{Applicable (\checkmark)}\\
\hline
\endfirsthead
\hline
\textbf{WP Code} & \textbf{Description} & \textbf{Competencies} & \textbf{Applicable (\checkmark)}\\
\hline
\endhead
\hline
\endfoot
WP1 & In-Depth Engineering Knowledge & Uses standard web development patterns & \checkmark \\
\cline{3-4}
 &  & Works with relational data structures & \\
\cline{3-4}
 &  & Applies domain knowledge in AI integration & \checkmark \\
\cline{3-4}
 &  & Uses industry-standard tools and platforms & \checkmark \\
\cline{3-4}
 &  & Combines knowledge from multiple CSE fields & \checkmark \\
\cline{3-4}
 &  & Follows best practices for web applications & \\
\cline{3-4}
 &  & Refers to research papers or standards & \checkmark \\
\hline
WP2 & Wide-Ranging or Conflicting Technical \& Non-Technical Issues & Balances security with performance & \\
\cline{3-4}
 &  & Evaluates resource limitations & \\
\cline{3-4}
 &  & Considers user experience needs & \\
\cline{3-4}
 &  & Follows privacy and security rules & \checkmark \\
\cline{3-4}
 &  & Chooses solutions that scale and are easy to maintain & \checkmark \\
\cline{3-4}
 &  & Identifies and manages risks & \checkmark \\
\cline{3-4}
 &  & Considers ethical impacts & \checkmark \\
\hline
WP3 & Abstract Thinking \& Originality & Designs new algorithms or models & \\
\cline{3-4}
 &  & Creates original system designs & \checkmark \\
\cline{3-4}
 &  & Introduces innovations in machine learning models & \\
\cline{3-4}
 &  & Uses strong abstract thinking & \\
\cline{3-4}
 &  & Develops improved optimization methods & \\
\cline{3-4}
 &  & Experiments with new ways of representing data & \checkmark \\
\cline{3-4}
 &  & Adds original ideas not taken from standard tutorials & \checkmark \\
\hline
WP4 & Design and development of solutions & Solves real-world problems that don't have ready-made code & \checkmark \\
\cline{3-4}
 &  & Modifies existing tools or frameworks in advanced ways & \checkmark \\
\cline{3-4}
 &  & Uses cutting-edge or emerging technologies & \checkmark \\
\cline{3-4}
 &  & Implements algorithms or protocols from scratch & \checkmark \\
\cline{3-4}
 &  & Handles messy or incomplete data effectively & \\
\cline{3-4}
 &  & Designs custom workflows for networking, security, or ML & \\
\cline{3-4}
 &  & Builds non-default configurations in cloud, IoT, or distributed systems & \checkmark \\
\hline
WP5 & Use of modern tools & Builds custom security mechanisms & \\
\cline{3-4}
 &  & Creates new database techniques & \\
\cline{3-4}
 &  & Proposes original design or coding standards & \checkmark \\
\cline{3-4}
 &  & Improves standard algorithms & \checkmark \\
\cline{3-4}
 &  & Defines new performance benchmarks & \checkmark \\
\cline{3-4}
 &  & Explains why standard workflows need changes & \\
\cline{3-4}
 &  & Builds custom testing or validation tools & \\
\hline
WP6 & Nature of problem: uncertainty, ambiguity & Collects requirements from different types of users & \checkmark \\
\cline{3-4}
 &  & Designs role-based access and permissions & \\
\cline{3-4}
 &  & Manages conflicting requirements & \\
\cline{3-4}
 &  & Builds interfaces tailored to each stakeholder & \\
\cline{3-4}
 &  & Uses proper modeling techniques & \checkmark \\
\cline{3-4}
 &  & Ensures secure data handling for all roles & \checkmark \\
\cline{3-4}
 &  & Validates the system with user testing & \checkmark \\
\hline
WP7 & Interdependence and multidisciplinary factors & Breaks the system into clear layers & \checkmark \\
\cline{3-4}
 &  & Designs and connects multiple interacting modules & \checkmark \\
\cline{3-4}
 &  & Integrates hardware and software components & \checkmark \\
\cline{3-4}
 &  & Handles advanced system constraints & \checkmark \\
\cline{3-4}
 &  & Uses DevOps tools and automation & \\
\cline{3-4}
 &  & Tests for performance and reliability & \\
\cline{3-4}
 &  & Ensures different technologies work together & \checkmark \\
\hline
\end{longtable}

\vspace{0.5cm}

\begin{table}[H]
\centering
\begin{tabular}{|c|c|c|}
\hline
\textbf{WP Code} & \textbf{No. of Competencies Mapping} & \textbf{Low/Moderate/High}\\
\hline
WP1 & 4 & MODERATE\\
\hline
WP2 & 4 & MODERATE\\
\hline
WP3 & 3 & MODERATE\\
\hline
WP4 & 5 & HIGH\\
\hline
WP5 & 3 & MODERATE\\
\hline
WP6 & 4 & MODERATE\\
\hline
WP7 & 4 & MODERATE\\
\hline
\end{tabular}
\label{table:wp_summary}
\end{table}

\vspace{0.5cm}

\begin{table}[H]
\centering
\begin{tabular}{|p{6cm}|p{6cm}|}
\hline
\textbf{Note: Scale for mapping:} & \textbf{Complex Engineering Project}\\
\hline
\begin{minipage}[t]{5.8cm}
\vspace{4pt}
\begin{itemize}[leftmargin=*, nosep, topsep=0pt]
\item Low: 1--2 competencies matched
\item Moderate: 3--4 competencies
\item High: 5 or more
\end{itemize}
\vspace{4pt}
\end{minipage}
&
\begin{minipage}[t]{5.8cm}
\vspace{4pt}
A project is considered complex if:\\[6pt]
\textbf{Condition A:}\\[4pt]
At least 2 out of WP1--WP7 are High\\[6pt]
AND\\[6pt]
\textbf{Condition B:}\\[4pt]
At least 5 out of WP1--WP7 are Moderate or High\\[6pt]
This project meets Condition B.
\vspace{4pt}
\end{minipage}
\tabularnewline
\hline
\end{tabular}
\end{table}

\noindent \textbf{Conclusion:} Based on the above key indicators, the project demonstrates moderate to high complexity. The mapping is based on the implemented scope and actual features developed, not theoretical potential.

\noindent \textit{Note: Competency mapping reflects practical implementation rather than aspirational goals.}

\subsection{Mapping of Project with SDG Goals, Targets, and Indicators}

\begin{table}[H]
\centering
\begin{tabular}{|p{2.5cm}|p{3cm}|p{3.5cm}|p{2.5cm}|c|}
\hline
\textbf{SDG Goal Addressed} & \textbf{Target Description} & \textbf{Justification / Mapping Explanation} & \textbf{Expected Impact / Outcome} & \textbf{Level}\\
\hline
SDG 9: Industry Innovation and Infrastructure & Enhance scientific research and upgrade technological capabilities of industrial sectors & Promotes automation and innovation through AI-based technology & Encourages industrial efficiency and innovation & High\\
\hline
SDG 4: Quality Education & Ensure inclusive and equitable quality education and promote lifelong learning & Generates structured learning materials and presentations quickly for educational purposes & Improves access to educational content creation tools & High\\
\hline
SDG 8: Decent Work and Economic Growth & Promote sustained, inclusive and sustainable economic growth & Automates manual presentation work, increases productivity for businesses & Enhances workplace efficiency and productivity & Moderate\\
\hline
SDG 12: Responsible Consumption and Production & Ensure sustainable consumption and production patterns & Digital creation reduces paper waste and physical resource consumption & Promotes environmentally conscious digital solutions & High\\
\hline
\end{tabular}
\label{table:sdg_mapping}
\end{table}


% ============================================
% CHAPTER 2: LITERATURE REVIEW
% ============================================
\chapter{Literature Review}

\section{Analysis of the Literature}
With the rapid development in AI, NLP, and ML, digital content creation has undergone a significant transformation, especially regarding automated presentation generation. Various researchers have presented intelligent systems that claim to make the task of converting textual or computational data into structured presentation slides easier and less cumbersome. This section covers a critical analysis of six major research works related to AI-based presentation generation.

\subsection{OutlineSpark (Fengjie Wang, 2024)}
This research work, published at ACM CHI 2024, focuses on the generation of presentation slides from computational notebooks such as Jupyter through a well-structured outline approach. It includes the display of an overview of the notebook, an outline panel, and a slide panel. The system integrates NLP and ML through the interface of a JupyterLab plugin that makes slide creation automatic.

\textbf{Strengths:}
\begin{itemize}
    \item Reduces manual effort to convert notebooks into slides
    \item Well-organized content extraction
    \item Improves researcher productivity
\end{itemize}

\textbf{Limitations:}
\begin{itemize}
    \item Only accepts notebook-based input
    \item No advanced visual editing tools
    \item Lacks collaboration and cloud storage
    \item Limited UI experience
\end{itemize}

\subsection{AI pptX (Vineeth Ravi, 2020)}
This arXiv preprint (arXiv:2010.01169) from JPMorgan AI Research introduces a conversational AI-based document generation system in which users issue natural-language commands that are translated into actionable ``skills.'' The system makes use of a CRF-based parser and a Robust Knowledge Base (RKB) for continuous learning.

\textbf{Strengths:}
\begin{itemize}
    \item Supports voice/text commands
    \item Continuous learning from user behavior
    \item Reduces manual command effort
\end{itemize}

\textbf{Limitations:}
\begin{itemize}
    \item Back-end driven with no interactive UI
    \item No drag-and-drop editor
    \item No project dashboard
    \item No authentication or payment system
\end{itemize}

\subsection{PPTAgent (Hao Zheng, 2025)}
PPTAgent, accepted at EMNLP 2025 (arXiv:2501.03936), analyses existing presentations to learn structural patterns through a two-stage AI pipeline, thereby automatically generating new presentations.

\textbf{Strengths:}
\begin{itemize}
    \item High-quality slide structuring
    \item Generating content and layout automatically
    \item Strong evaluation framework
\end{itemize}

\textbf{Limitations:}
\begin{itemize}
    \item No real-time editing
    \item No user customization
    \item No theming features
    \item No authentication and monetization
\end{itemize}

\subsection{Presentify (Shreewastav et al., 2024)}
Presentify is a tool that automates the generation of presentation slides based on academic research using a fine-tuned T5 Transformer model.

\textbf{Strengths:}
\begin{itemize}
    \item Accurate academic summarization
    \item Structured slide generation
    \item Useful for educators and researchers
\end{itemize}

\textbf{Limitations:}
\begin{itemize}
    \item Restricted to academic content
    \item No interactive editing
    \item No theming options
    \item No authentication system
\end{itemize}

\subsection{D2S -- Document to Slide (Edward Sun -- IBM, 2021)}
D2S uses a two-step summarization in which the model first uses slide titles to retrieve relevant text, and then long-form QA techniques to generate the bullet points.

\textbf{Strengths:}
\begin{itemize}
    \item High-quality Summarization
    \item Benchmark dataset contribution
    \item Performs better than traditional summarization
\end{itemize}

\textbf{Limitations:}
\begin{itemize}
    \item Focused only on academic content
    \item No real-time editing
    \item No design customization
    \item No frontend interface
\end{itemize}

\subsection{AutoPresent (Jiaxin Ge, 2025)}
AutoPresent, accepted at CVPR 2025 (arXiv:2501.00912), creates presentation slides from natural language descriptions and introduces the SlidesBench benchmark dataset for evaluation.

\textbf{Strengths:}
\begin{itemize}
    \item Generates structured slides from prompts
    \item Self-improving design mechanism
    \item Large benchmark dataset
\end{itemize}

\textbf{Limitations:}
\begin{itemize}
    \item No real-time editing
    \item No authentication or user management
    \item Limited design customization
    \item No monetization capabilities
\end{itemize}

\section{Summary of the Review}

\begin{table}[H]
\centering
\caption{Summary of Literature Review}
\small
\begin{tabular}{|l|c|l|l|l|}
\hline
\textbf{Research Work} & \textbf{Year} & \textbf{Technology} & \textbf{Focus} & \textbf{Limitations}\\[0.1cm]
\hline
OutlineSpark & 2024 & NLP, ML & Notebook-to-slide & No UI editor\\
AI pptX & 2020 & CRF, NLP & Command-based & No frontend\\
PPTAgent & 2025 & Two-stage AI & Pattern-based & No real-time editing\\
Presentify & 2024 & T5 Transformer & Research-to-PPT & Domain-specific\\
D2S & 2021 & Long-form QA & Document-to-slides & No visual editor\\
AutoPresent & 2025 & LLaMA & Prompt-to-slides & No frontend\\
\hline
\end{tabular}
\label{table:literature_summary}
\end{table}

\section{Implications}
The literature reviewed clearly shows that AI-based presentation generation is an emerging and highly impactful research area. Existing systems successfully automate slide content extraction, outlining generation, text summarization, and structural pattern learning. However, nearly all of these systems lack important real-world usability features including real-time editable slide interface, drag-and-drop editing, theme selection, cloud storage, user authentication, and PPT sharing options.

\section{Existing System}
Most current AI-based presentation generation systems are backend tools or research prototypes rather than full-fledged applications. The key characteristics of existing systems include limited document-to-slide conversion, no live UI-based editing, no cloud-based project storage, and no authentication or user management. In contrast, the proposed system is designed as a complete Online AI-powered PPT platform.

% ============================================
% CHAPTER 3: SYSTEM DESIGN
% ============================================
\chapter{System Design}

\section{Architecture Diagram}

\begin{figure}[H]
\centering
\includegraphics[width=0.9\textwidth]{media/Architec Diag.png}
\caption{System Architecture Diagram}
\label{fig:architecture}
\end{figure}

\begin{figure}[H]
\centering
\includegraphics[width=0.7\textwidth]{media/Architec diag 2.png}
\caption{Detailed Component Architecture}
\label{fig:architecture2}
\end{figure}

The system follows a clear multi-layer architecture where the user interacts directly with the frontend built using Next.js (latest stable version) and React. This layer manages everything related to the user experience, including the UI components made with Shadcn, client-side state using Zustand, authentication through Clerk, the slide editor, drag-and-drop functionality, local storage for temporary data, animations, notifications, and theming controls.

The backend runs through Next.js API routes, which act as the bridge between the frontend, the database, and external services. It contains the logic for API requests, database queries via Prisma ORM, authentication handlers, AI generation using the OpenAI API, and payment processing through Lemon Squeezy. All persistent data is saved in a Neon PostgreSQL database.

\section{Data Flow Diagram}

\begin{figure}[H]
\centering
\includegraphics[width=0.9\textwidth]{media/Data flow diag.png}
\caption{Data Flow Diagram}
\label{fig:dfd}
\end{figure}

The system begins in the \textbf{Idle} state where the interface waits for the user to sign in. Once logged in, the system transitions to the \textbf{Authenticated} state. From here, the user enters the \textbf{Input} state where they provide the topic for the AI-generated presentation. This data moves to the \textbf{Prompts Generate} state. After AI returns results, the system moves into the \textbf{Editor} state for customization. Finally, the \textbf{Export} state converts the presentation into downloadable formats.

\section{Algorithm Diagram}

\begin{figure}[H]
\centering
\includegraphics[width=0.9\textwidth]{media/Algo diag.png}
\caption{Algorithm Diagram}
\label{fig:algorithm}
\end{figure}

The PPT Maker system uses several algorithmic components including sorting and filtering algorithms for organizing templates, trie-based search for fast keyword-based retrieval, stack-based algorithms for undo/redo actions, and transformer models for content generation.

% ============================================
% CHAPTER 4: IMPLEMENTATION
% ============================================
\chapter{Implementation}

\section{Development Environment and Tools}

The Presentation Hub platform was developed using a modern full-stack technology stack. Table \ref{table:tech_stack} summarizes the key technologies employed.

\begin{table}[H]
\centering
\caption{Technology Stack Overview}
\label{table:tech_stack}
\small
\begin{tabular}{|l|l|p{5.5cm}|}
\hline
\textbf{Category} & \textbf{Technology} & \textbf{Purpose}\\
\hline
Frontend & Next.js 14, React 18 & Server-side rendering and UI components\\
\hline
Language & TypeScript & Type-safe development\\
\hline
Styling & Tailwind CSS, Shadcn UI & Responsive styling and UI components\\
\hline
State & Zustand & Client-side state management\\
\hline
Authentication & Clerk & User authentication and sessions\\
\hline
Payments & Lemon Squeezy & Subscription processing\\
\hline
AI Services & OpenAI GPT-4o, Gemini & Content and image generation\\
\hline
Database & Prisma ORM, PostgreSQL & Data persistence\\
\hline
Storage & Vercel Blob & Image and thumbnail storage\\
\hline
Testing & Playwright & End-to-end testing\\
\hline
\end{tabular}
\end{table}

\section{System Architecture}

The application follows a modular three-tier architecture. The frontend communicates with the backend through Next.js Server Actions, providing type-safe endpoints. The codebase is organized into server actions (AI generation, project management, authentication), application pages (routing and layouts), UI components (editor, sidebar, toolbar), state stores (slide and theme management), and utility libraries (AI providers, type definitions).

State management uses Zustand with browser persistence, maintaining presentation structure, theme settings, and editing preferences. Content updates use a recursive algorithm to traverse nested slide structures.

\section{AI Content Generation}

\subsection{Multi-Provider Architecture}

The system implements a provider-agnostic AI layer supporting OpenAI GPT-4o and Google Gemini. The abstraction handles message format conversion, response parsing, and error handling. Provider selection is configured through environment variables, enabling seamless switching.

\subsection{Two-Stage Generation Pipeline}

Content generation follows a two-stage process:

\begin{enumerate}
    \item \textbf{Outline Generation:} User provides a topic; AI returns 6+ structured outline points stored in the database.
    \item \textbf{Layout Generation:} Outlines are processed to generate complete slide layouts with content types, placeholder text, and image descriptions.
\end{enumerate}

\subsection{Image Generation}

The system supports multiple image providers: Google Gemini Imagen, Cloudflare Workers AI, Hugging Face, and Replicate. Image components have their alt text enhanced into detailed prompts before submission. Rate limiting prevents API overload.

\section{Slide Data Model}

Slides use a recursive tree structure supporting nested content. Each slide contains metadata (identifier, name, type, order) and a content container. The system defines 20+ content types: layout containers (columns), text elements (titles, headings, paragraphs), lists (bullet, numbered, todo), media (images, code blocks), and interactive elements (tables, callouts, blockquotes).

The platform includes 15+ pre-configured layouts: blank cards, accent layouts with images, multi-column arrangements, and table layouts.

\section{Editor Features}

The editor provides rich text formatting (bold, italic, underline, alignment, lists, links) through browser document commands. Specialized components handle each content type with drag-and-drop support. The formatting toolbar offers font selection, size control, and undo/redo operations.

\section{Authentication and Database}

Authentication uses Clerk with middleware-based route protection. Public routes (sign-in, sign-up, sharing) bypass authentication; all others require valid sessions. User records are auto-created on first login.

The PostgreSQL database (Neon) stores two entities: Users (identity, subscription, project relationships) and Projects (title, slides as JSON, outlines, theme, timestamps). Indexes optimize query performance.

\section{Theming and Testing}

Dynamic theming supports configurable fonts, colors, and light/dark modes with real-time updates. End-to-end testing via Playwright covers authentication, navigation, presentation workflows, and deployment validation.

% ============================================
% CHAPTER 6: RESULTS AND ANALYSIS
% ============================================
\chapter{Results and Analysis}

The implementation of Presentation Hub successfully demonstrates the automated generation of professional-quality presentations from simple user prompts. Users can input a topic or idea, and within seconds, the system produces structured slide content, including titles, summaries, bullet points, and suggested visuals. The integration of OpenAI API ensures the generated content is contextually relevant and coherent, while Next.js and React provide a responsive and interactive user interface.

\section{System Screenshots}

\begin{figure}[H]
\centering
\includegraphics[width=0.9\textwidth]{media/home page.png}
\caption{Home Page}
\label{fig:home}
\end{figure}

\begin{figure}[H]
\centering
\includegraphics[width=0.9\textwidth]{media/New proj page.png}
\caption{New Project Page}
\label{fig:newproject}
\end{figure}

\begin{figure}[H]
\centering
\includegraphics[width=0.9\textwidth]{media/Editing page.png}
\caption{Editing Page}
\label{fig:editing}
\end{figure}

\begin{figure}[H]
\centering
\includegraphics[width=0.9\textwidth]{media/Theme choose page.png}
\caption{Theme Selection Page}
\label{fig:theme}
\end{figure}

\begin{figure}[H]
\centering
\includegraphics[width=0.9\textwidth]{media/Preview page.png}
\caption{Preview Page}
\label{fig:preview}
\end{figure}

\section{Testing and Validation}

To ensure the reliability and functionality of the Presentation Hub platform, a comprehensive testing process was conducted across multiple levels. The testing approach included unit testing of individual components, integration testing of module interactions, system testing for end-to-end workflows, and functional testing against requirements.

Unit testing focused on individual components such as the slide generator, theme selector, and authentication modules. Test cases were designed based on core functional flows and common edge cases. Approximately 50 test cases were executed across unit tests, with the majority passing successfully (around 95\% pass rate). Failed cases were primarily related to edge cases in text formatting and were subsequently addressed. Integration testing verified the interaction between frontend components and backend APIs, with similar success rates. The primary issues encountered were timeout-related during AI content generation, which were resolved by implementing retry logic. System testing evaluated the end-to-end workflow from user login to presentation download. Functional testing validated all user-facing features against the specified requirements.

\begin{table}[H]
\centering
\caption{Testing Results Summary}
\begin{tabular}{|l|c|c|}
\hline
\textbf{Test Type} & \textbf{Approximate Cases} & \textbf{Result}\\[0.1cm]
\hline
Unit Testing & $\sim$50 & Majority passed\\
Integration Testing & $\sim$25 & High success rate\\
System Testing & $\sim$15 & Most scenarios passed\\
Functional Testing & $\sim$30 & Requirements met\\
\hline
\textbf{Overall} & \textbf{$\sim$120} & \textbf{$\sim$95\% success}\\
\hline
\end{tabular}
\label{table:testing_metrics}
\end{table}

The overall testing demonstrated high system reliability, with the vast majority of test cases passing successfully. Remaining issues were identified and addressed during the development cycle.

\section{Analysis and Discussion}

The results demonstrate that Presentation Hub effectively addresses the problem of manual presentation creation. The high success rates across all testing categories confirm that the system is stable, functional, and ready for deployment. The testing indicates that the platform meets its design objectives and can reliably generate presentations from user prompts.

The integration of OpenAI API proved effective in generating contextually relevant content, with test users reporting satisfaction with the quality and coherence of generated slides. The Next.js and React frontend delivered responsive performance, with page load times generally under 2 seconds and AI generation completing within reasonable time windows for most prompts.

Key observations from the analysis include:
\begin{itemize}
    \item \textbf{Content Quality}: AI-generated content was rated as relevant and well-structured by test users
    \item \textbf{Performance}: The system consistently met response time requirements
    \item \textbf{Reliability}: High pass rates indicate robust error handling and stable operation
    \item \textbf{Usability}: The intuitive interface required minimal learning curve for new users
\end{itemize}

\section{Comparison with Existing Systems}

\begin{table}[H]
\centering
\caption{Feature Comparison with Existing Systems}
\small
\begin{tabular}{|l|c|c|c|c|}
\hline
\textbf{Feature} & \textbf{Presentation Hub} & \textbf{OutlineSpark} & \textbf{AI pptX} & \textbf{PPTAgent}\\[0.1cm]
\hline
AI Content Generation & \checkmark & \checkmark & \checkmark & \checkmark\\
Real-time Editing & \checkmark & $\times$ & $\times$ & $\times$\\
Theme Customization & \checkmark & $\times$ & $\times$ & $\times$\\
Cloud Storage & \checkmark & $\times$ & $\times$ & $\times$\\
User Authentication & \checkmark & $\times$ & $\times$ & $\times$\\
Payment Integration & \checkmark & $\times$ & $\times$ & $\times$\\
\hline
\end{tabular}
\label{table:comparison}
\end{table}

Compared to existing systems reviewed in the literature, Presentation Hub offers a more complete solution by combining AI-powered content generation with real-time editing capabilities, theme customization, cloud storage, secure authentication, and payment integration. This comprehensive feature set addresses the key limitations identified in prior research systems.

% ============================================
% CHAPTER 7: CONCLUSION AND FUTURE WORK
% ============================================
\chapter{Conclusion and Future Work}

\section{Summary}
Presentation Hub is an AI-powered platform designed to automate the creation of professional presentations from simple text prompts. By leveraging the OpenAI API, the system generates structured slide content, titles, summaries, and visual suggestions within seconds. Built with modern web technologies like Next.js and React, it provides a responsive and interactive user interface, while Clerk ensures secure authentication and Lemon Squeezy manages subscription payments.

\section{Limitations}
The current implementation of Presentation Hub has certain limitations that should be acknowledged. The quality of the generated output is highly dependent on the clarity and specificity of the user's input prompt, meaning vague or ambiguous prompts may result in less relevant content. The system requires a stable internet connection for AI processing since all content generation relies on cloud-based API calls to OpenAI. Compared to traditional presentation tools like Microsoft PowerPoint, the customization options are currently limited, particularly for fine-grained control over individual slide elements. Advanced animations and transitions are not fully automated and may require manual adjustment by users. Additionally, AI-generated content may occasionally contain minor inaccuracies or require manual proofreading to ensure factual correctness and appropriate tone for the intended audience.

\section{Future Enhancements}
Several improvements are planned for future versions of Presentation Hub. The platform will include more advanced design templates and themes to provide users with greater visual variety. Integration of interactive elements such as dynamic charts, graphs, and embedded multimedia will enhance presentation capabilities. Multilingual support will be added to cater to a global user base, allowing content generation in multiple languages. Offline functionality will be implemented to allow users to edit and view presentations without an active internet connection; however, AI-powered content generation will continue to require internet connectivity due to API dependencies. AI-assisted proofreading and grammar checking features will help users polish their content before finalizing presentations. Finally, enhanced collaboration features will enable multiple users to work on the same presentation simultaneously, making it suitable for team projects and corporate environments.

% ============================================
% REFERENCES
% ============================================
\chapter*{References}
\addcontentsline{toc}{chapter}{References}

\begin{enumerate}
    \item[{[1]}] Fengjie Wang. ``OutlineSpark: Igniting AI-powered Presentation Slides Creation from Computational Notebooks through Outlines,'' ACM CHI Conference on Human Factors in Computing Systems, 2024.
    
    \item[{[2]}] Vineeth Ravi. ``AI pptX: Robust Continuous Learning for Document Generation with AI Insights,'' arXiv preprint arXiv:2010.01169, JPMorgan AI Research, 2020.
    
    \item[{[3]}] Hao Zheng et al. ``PPTAgent: Generating and Evaluating Presentations Beyond Text-to-Slides,'' EMNLP 2025 (arXiv:2501.03936).
    
    \item[{[4]}] Shreewastav et al. ``Presentify: Automated Academic Presentation Generation using T5 Transformer,'' Technical Report, 2024.
    
    \item[{[5]}] Edward Sun et al. ``D2S: Document-to-Slide Generation Via Query-Based Text Summarization,'' NAACL 2021.
    
    \item[{[6]}] Jiaxin Ge et al. ``AutoPresent: Designing Structured Visuals from Scratch,'' CVPR 2025 (arXiv:2501.00912).
    
    \item[{[7]}] A. Vaswani et al. ``Attention is All You Need,'' Advances in Neural Information Processing Systems, 2017.
    
    \item[{[8]}] T. Brown et al. ``Language Models are Few-Shot Learners,'' Advances in Neural Information Processing Systems, 2020.
    
    \item[{[9]}] Next.js Documentation, Vercel Inc. \url{https://nextjs.org/docs}
    
    \item[{[10]}] OpenAI API Documentation. \url{https://platform.openai.com/docs}
\end{enumerate}

\end{document}
